% !TEX TS-program = pdflatex
% !TEX encoding = UTF-8 Unicode

% This is a simple template for a LaTeX document using the "article" class.
% See "book", "report", "letter" for other types of document.

\documentclass[11pt]{article} % use larger type; default would be 10pt

\usepackage[utf8]{inputenc} % set input encoding (not needed with XeLaTeX)

%%% Examples of Article customizations
% These packages are optional, depending whether you want the features they provide.
% See the LaTeX Companion or other references for full information.

%%% PAGE DIMENSIONS
\usepackage{geometry} % to change the page dimensions
\geometry{a4paper} % or letterpaper (US) or a5paper or....
% \geometry{margin=2in} % for example, change the margins to 2 inches all round
% \geometry{landscape} % set up the page for landscape
%   read geometry.pdf for detailed page layout information

\usepackage{graphicx} % support the \includegraphics command and options

% \usepackage[parfill]{parskip} % Activate to begin paragraphs with an empty line rather than an indent

%%% PACKAGES
\usepackage{booktabs} % for much better looking tables
\usepackage{array} % for better arrays (eg matrices) in maths
\usepackage{paralist} % very flexible & customisable lists (eg. enumerate/itemize, etc.)
\usepackage{verbatim} % adds environment for commenting out blocks of text & for better verbatim
\usepackage{subfig} % make it possible to include more than one captioned figure/table in a single float
% These packages are all incorporated in the memoir class to one degree or another...

%%% HEADERS & FOOTERS
\usepackage{fancyhdr} % This should be set AFTER setting up the page geometry
\pagestyle{fancy} % options: empty , plain , fancy
\renewcommand{\headrulewidth}{0pt} % customise the layout...
\lhead{}\chead{}\rhead{}
\lfoot{}\cfoot{\thepage}\rfoot{}

%%% SECTION TITLE APPEARANCE
\usepackage{sectsty}
\allsectionsfont{\sffamily\mdseries\upshape} % (See the fntguide.pdf for font help)
% (This matches ConTeXt defaults)

%%% ToC (table of contents) APPEARANCE
\usepackage[nottoc,notlof,notlot]{tocbibind} % Put the bibliography in the ToC
\usepackage[titles,subfigure]{tocloft} % Alter the style of the Table of Contents
\renewcommand{\cftsecfont}{\rmfamily\mdseries\upshape}
\renewcommand{\cftsecpagefont}{\rmfamily\mdseries\upshape} % No bold!

%%% END Article customizations

%%% The "real" document content comes below...

\title{Brief Article}
\author{The Author}
%\date{} % Activate to display a given date or no date (if empty),
         % otherwise the current date is printed 

\begin{document}
In this section, we focus on the case of a single drone, i.e., \(k=1\), and we show that a simple greedy hill-climbing approach (Algorithm~\ref{alg:submodular-greedy}) gives a constant factor approximation to the problem of maximizing \prob.
To give a lower bound on the approximation ratio of Algorithm~\ref{alg:submodular-greedy}, we show that the objective function $\mathcal{P}(S)$ is \emph{monotone and submodular}: For a ground set $N$, a function $z:2^N\rightarrow \R$ is submodular if for any pair of sets $S\subseteq T \subseteq N$ and for any element $e\in N\setminus T$, $z(S\cup\{e\}) - z(S) \geq z(T\cup \{e\}) - z(T)$.
Then, since we have an additive cost constraints function, we can exploit a known result on submodular function optimization~\cite{sviridenko2004note}.
Given a finite set $N$, an integer $k'$, two non-decreasing real-valued functions $z$ and $c$ defined on the set of subsets of $N$, the problem of finding a set $S\subseteq N$ such that $c(S)\leq k'$ and $z(S)$ is maximum can be $1-\frac{1}{e}$ approximated by starting with the empty set and repeatedly adding the element that maximizes the ratio between the reward gained and the cost, i.e., \(\frac{z(S\cup\{x\}) - z(S)}{c(\{x\})}\), if $z$ is monotone and submodular.
%\federico{forse e' venuta un po' ``brodosa''. Potremmo pensare di tagliare/accorciare/spostare queste cose sopra.}


\begin{theorem}
Function \(\mathcal{P}(S)\) is monotone and submodular with respect to any feasible solution for \prob.
\end{theorem}
\begin{proof}

Let us first note that, given a solution \(S\), it is not possible to add an interval \(\bar{I}\) if it is not compatible with at least one other interval in \(S\), i.e., \(\exists I_j \in S\text{ s.t. } \bar{I} \cap I_j \neq \emptyset\).
Thus, if \(\bar{I}\) is not compatible \(\mathcal{P}(S \cup \{\bar{I}\}) = \mathcal{P}(S)\) since it can not be added to the solution.
Moreover, in the following we make use of the observation that, by definition, \(\mathcal{P}(S\cup\{\bar{I}\}) = \mathcal{P}(S) + p_{\bar{I}}\) for any set \(S\) and interval \(\bar{I}\).


To prove that \(\mathcal{P}(S)\) is monotone, we prove that for each interval \(\bar{I}\in I\) and solution \(S \subseteq I\) we have 
\(
\mathcal{P}(S) \le \mathcal{P}(S \cup \{\bar{I}\})
\).


We first notice that there are two possible cases depending on the compatibility of interval \(\bar{I}\) w.r.t. to the solution set \(S\).
Namely, if there exists an interval \(I_j \in S\) such that the intersection is not empty, i.e., \(\bar{I} \cap I_j \neq \emptyset\),then we have
\( \mathcal{P}(S) \le \mathcal{P}(S \cup \{\bar{I}\}) = \mathcal{P}(S)\).
Otherwise we have that
% \begin{align*}
% \mathcal{P}(S) &\le \mathcal{P}(S \cup \{\bar{I}\}) = \mathcal{P}(S) + p_{\bar{I}}
% \\
% 0 &\le p_{\bar{I}}
% \end{align*}
\(
\mathcal{P}(S) \le \mathcal{P}(S \cup \{\bar{I}\}) = \mathcal{P}(S) + p_{\bar{I}}
\), thus \(0 \le p_{\bar{I}}\) which is always true.


To prove that \(\mathcal{P}(S)\) is submodular, we prove that for any two solutions \(S,T\) such that \(S\subseteq T\) and for any interval \(I_j \in I\setminus T\), it holds:
\[
 \mathcal{P}(S\cup\{\bar{I}\}) - \mathcal{P}(S) \geq \mathcal{P}(T\cup \{\bar{I}\}) - \mathcal{P}(T).
\]

We analyze the following cases:
\begin{itemize}
\item If there not exists an interval \(I_j \in T\) such that \(\bar{I} \cap I_j \neq \emptyset\). 
% \begin{align*}
% \mathcal{P}(S\cup\{\bar{I}\}) - \mathcal{P}(S) &\geq \mathcal{P}(T\cup \{\bar{I}\}) - \mathcal{P}(T)
% \\
% p_{\bar{I}} &\geq p_{\bar{I}}
% \end{align*}
\(
\mathcal{P}(S\cup\{\bar{I}\}) - \mathcal{P}(S) \geq \mathcal{P}(T\cup \{\bar{I}\}) - \mathcal{P}(T)
\), thus \(p_{\bar{I}} \geq p_{\bar{I}}\) which is always true.

\item If there not exists an interval  \(u \in T\setminus S\) such that \(\bar{I} \cap I_j \neq \emptyset\)
% \begin{align*}
% \mathcal{P}(S\cup\{\bar{I}\}) - \mathcal{P}(S) &\geq \mathcal{P}(T\cup \{\bar{I}\}) - \mathcal{P}(T)
% \\
% p_{\bar{I}} \geq 0
% \end{align*}
\(
\mathcal{P}(S\cup\{\bar{I}\}) - \mathcal{P}(S) \geq \mathcal{P}(T\cup \{\bar{I}\}) - \mathcal{P}(T)
\), thus \(p_{\bar{I}} \geq 0\) which is always true.

\item If there not exists an interval  \(u \in S\) such that \(\bar{I} \cap I_j \neq \emptyset\)
\(
0 = \mathcal{P}(S\cup\{\bar{I}\}) - \mathcal{P}(S) \geq \mathcal{P}(T\cup \{\bar{I}\}) - \mathcal{P}(T) = 0
\)
\end{itemize}
\end{proof}

\begin{corollary}
Algorithm~\ref{alg:submodular-greedy} provides a $\left(1-\frac{1}{e}\right)$-approximation for \prob.
\end{corollary}

\begin{algorithm}[ht]
	$S \gets \emptyset$\;
	\While{$ c(S) \le B $} {
		$\bar{I} \gets \argmax_{I_i \in I \setminus S} \frac{p_i}{c(v_i)}$\;
		
		\If{$ c(S \cup \bar{I}) \le B \land \bar{I} \cap I_j = \emptyset\; \forall I_j \in S$ } {
			$S \gets S \cup \bar{I}$\;
		}
	}
	\Return{$S$}
	\caption{\apxs}
	\label{alg:submodular-greedy}
\end{algorithm}

\end{document}
